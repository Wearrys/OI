\documentclass[ignorenonframetext,]{beamer}
\setbeamertemplate{caption}[numbered]
\setbeamertemplate{caption label separator}{: }
\setbeamercolor{caption name}{fg=normal text.fg}
\beamertemplatenavigationsymbolsempty
\usepackage{lmodern}
\usepackage{amssymb,amsmath}
\usepackage{ifxetex,ifluatex}
\usepackage{fixltx2e} % provides \textsubscript
\ifnum 0\ifxetex 1\fi\ifluatex 1\fi=0 % if pdftex
  \usepackage[T1]{fontenc}
  \usepackage[utf8]{inputenc}
\else % if luatex or xelatex
  \ifxetex
    \usepackage{fontspec}
    \usepackage[UTF8]{ctex}
    \usetheme{CambridgeUS}
    \usecolortheme{spruce}
    \useinnertheme{circles}
    \usefonttheme[onlymath]{serif}
  \else
    \usepackage{fontspec}
  \fi
  \defaultfontfeatures{Ligatures=TeX,Scale=MatchLowercase}
\fi
% use upquote if available, for straight quotes in verbatim environments
\IfFileExists{upquote.sty}{\usepackage{upquote}}{}
% use microtype if available
\IfFileExists{microtype.sty}{%
\usepackage{microtype}
\UseMicrotypeSet[protrusion]{basicmath} % disable protrusion for tt fonts
}{}
\newif\ifbibliography

% Prevent slide breaks in the middle of a paragraph:
\widowpenalties 1 10000
\raggedbottom

\AtBeginPart{
  \let\insertpartnumber\relax
  \let\partname\relax
  \frame{\partpage}
}
\AtBeginSection{
  \let\insertsectionnumber\relax
  \let\sectionname\relax
  %\frame{\sectionpage}
  \begin{frame}
    \tableofcontents[sectionstyle=show/shaded, hideallsubsections]
  \end{frame}
}
\AtBeginSubsection{
  \let\insertsubsectionnumber\relax
  \let\subsectionname\relax
  %\frame{\subsectionpage}
}

\setlength{\emergencystretch}{3em}  % prevent overfull lines
\providecommand{\tightlist}{%
  \setlength{\itemsep}{0pt}\setlength{\parskip}{0pt}}
\setcounter{secnumdepth}{0}

\title{计数技巧选讲}
\author{Wearry}
\date{Stay determined!}

\begin{document}
\frame{\titlepage}

\begin{frame}
\tableofcontents[hideallsubsections]
\end{frame}

\section{容斥原理}\label{ux5bb9ux65a5ux539fux7406}

\subsection{Calc}\label{calc}

\begin{frame}

\frametitle{\insertsubsection}

定义序列 \(\{a_i\}\) 是合法的当且仅当:

\begin{itemize}
\tightlist
\item
  \(\forall i ,\, a_i \in [1, C]\)
\item
  \(\forall i,j ,\, a_i \neq a_j\)
\end{itemize}

定义序列的权值为所有元素的乘积, 求所有长度为 \(n\)
的不同的合法序列的权值和, 对 \(10 \times 9 + 7\) 取模。

\vspace{2ex}

\(n \le 500, C \le 10^9\)

https://loj.ac/problem/2731

\end{frame}

\section{状态转移技巧}\label{ux72b6ux6001ux8f6cux79fbux6280ux5de7}

\begin{frame}

http://hihocoder.com/contest/challenge37/problem/4 pt d1t3
http://uoj.ac/problem/390

\end{frame}

\section{Prufer序列}\label{pruferux5e8fux5217}

\begin{frame}

http://hihocoder.com/contest/challenge28/problem/4
https://loj.ac/problem/2320

\end{frame}

\section{多项式理论}\label{ux591aux9879ux5f0fux7406ux8bba}

\begin{frame}

http://hihocoder.com/contest/challenge37/problem/3
https://loj.ac/problem/2504 pt d2t3

\end{frame}

\end{document}
